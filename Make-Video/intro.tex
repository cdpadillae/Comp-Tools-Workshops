\section{Introducción}
Durante la Segunda Guerra Mundial, el Germanio y el Silicio, que son semiconductores intrínsecos, fueron elementos de especial interés y estudio, puesto que eran usados para la fabricación de radares. Así pues, en los años de la post guerra permitieron la creación de diodos y, posteriormente, gracias a Bardeen, Brattain y Shockley, se descubrió el efecto transistor, y fue posible la creación del transistor de unión bipolar (BJT, por sus siglas en inglés).

El transistor se puede entender como la unión de tres semiconductores dopados, ya sean del tipo NPN o PNP, donde,  la primera región se conoce como \textit{emisor}, en el centro se encuentra la \textit{base}, y la última región es el \textit{colector}. Un ejemplo de configuración NPN se muestra en la figura, el ancho de la base es mucho menor comparado con las otras dos regiones.

Las dos junturas que se forman entre las regiones de los extremos y la base se conocen como \textit{juntura de emisor} y \textit{juntura de colector}. Los transistores se caracterizan por tener tres (3) terminales y se pueden representar en un circuito como se muestra en la figura. A pesar de que distintas configuraciones son posibles, para que funcione, el transistor debe polarizarse de manera directa en la juntura base-emisor y de manera inversa en la juntura base-colector.
En un transistor se consideran tres diferencias de potencial que son las que existen entre sus electrodos: Emisor-Colector $V_{EC}$, Colector-Base $V_{CB}$, Base-Emisor $V_{BE}$ tal que se cumple:
\begin{equation}
    V_{EC}+V_{CB}+V_{BE} = 0.
\end{equation}

Así mismo se consideran tres corrientes: base ($I_{B}$), emisor ($I_{E}$) y colector ($I_{C}$); que cumplen:
\begin{equation}
    I_{E}+I_{B}+I_{C} = 0
\end{equation}
Una representación de lo anterior se muestra en la figura para ambos tipos de transistor.

Como ejemplo para entender el efecto transistor se tomará un transistor NPN, similar al de la figura. Lo que ocurre cuando se conecta de manera directa la juntura base-emisor y de manera inversa base-colector ocurre lo siguiente:
\begin{itemize}
    \item Los portadores mayoritarios del material emisor son electrones. 
    \item Al estar polarizada de manera directa la juntura emisor-base los electrones pasan a la región de la base.
    \item Al ser la región de la base tan delgada, hay una recombinación por la presencia de los huecos pero la mayoría de electrones siguen libres.
    \item La juntura base-colector esta polarizada de manera inversa, así que los electrones que pasaron a la base son atraídos por la diferencia de potencial en el terminal del colector.
    \item Una gran cantidad de electrones en la región de la base cruzan la juntura base-colector y llegan a la región del colector, generando la corriente del colector.
\end{itemize}
El hecho de polarizar una juntura en directo y la otra en inverso fue lo que le dio el nombre de \textbf{Transistor de unión bipolar}, que fue el reemplazo de los tubos de vacío y años más tarde se usó para la invención de los circuitos integrados, éstos últimos han sido clave en el desarrollo tecnológicos de las últimas dos décadas. 
